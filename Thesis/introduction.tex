\section{Introduction}

This thesis focuses on the competitive co-evolution of neural controllers with the hopes of evolving complex behaviour in the game of predator vs prey. Played on a board sized 9x9, the agents move simultaneously in hopes that the predator captures the prey while the prey attempts to avoid the predator. To encourage evolution within the system, walled and non-walled boards were tried along with variations in game rules. These game rules were adapted as the system was tested in order to counteract undesirable developments in previous versions of the game, or to encourage certain behaviours in the new version. To much surprise, some unexpected advancements were made by the prey in the game. The prey tended to exploit the game rules to their advantage, while the predators were left behind competing with their less developed networks. 