
\section{Conclusion}

Throughout tis research, various parameter sets were tested with several variations of game rules for the predator vs prey game. It was found that the most balanced game rules occurred when the ability to fall off the map did not exist, as this forced the two types of agents to interact. Although the agents were forced to interact, due to the probability of the predator capturing the prey, the prey performed better while needing less development. As such the prey behaviour did not evolve to be as complex as the predator behaviour. The prey tended to get stuck in a corner and get caught by the predator. 

Game V1 was then introduced as a version of the game where there are no walls blocking the prey. With the walls not blocking the prey from leaving the board, the prey continues similarly to the walled version of the game, except the prey continues to walk off, and since the predator could not catch the prey, the game ends with the prey being the victor. This caused the prey to continuously run off the board as that is the easiest way to way the game for the prey.

In order to stop such a scenario, Game V2 implemented a penalty if a agent falls off the board. This initially provided the predators a better fitness, but as the evolution took place, the prey always ended up out evolving the predators. In this game version, both sides evolved a patrolling technique. This caused most games to end after 20 steps with the prey emerging victorious. 

In order to encourage the predator to catch the prey, a penalty was introduced to the predator if the prey ran off the edge in game V3. This was done in hopes that the predator would rush to capture the prey before the prey fell. However the prey ended up exploiting this new game rule. In most of the experiments, the prey score was high, and when looking at the games, the prey tends to run off the edge of the board. This stifles the learning of the predator enough that the prey end up outsmarting the predators.